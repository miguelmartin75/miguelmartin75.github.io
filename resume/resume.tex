\documentclass[10pt,oneside,a4paper]{article}

\usepackage{geometry}
\usepackage{multicol}
\usepackage{titlesec}
\usepackage{enumitem}
\usepackage{setspace}
\pagestyle{empty}

\geometry{legalpaper, hmargin=1in, vmargin=0.5in}

\setstretch{1.0}
\setlength{\parindent}{0pt}
\titlespacing*{\section}{0pt}{0.5em}{0pt}
\titlespacing*{\hrulefill}{0pt}{0pt}{2em}

\newcommand{\project}[5] 
{
    {
    \textbf{#1} \hfill {\footnotesize\textbf{#2}}\newline
    \emph{#3} \hfill \emph{#4}
    }

    {
        \begin{description}[leftmargin=0.5cm, itemindent=0cm]
        \item {#5}
        \end{description}
    }
}

\newcommand{\info}[2] 
{
    {
        \section*{#1}
        \hrulefill\newline
    }

    {#2}
}

\begin{document}

{
    \raggedleft
    \footnotesize
    \textbf{Last Modified:} \today

    \centering
    {\huge\textbf{Miguel Martin}}

    \textbf{Email:} miguel@miguel-martin.com\hfill\textbf{Website:} miguel-martin.com\newline

}

\info{Experience}
{
    {
        \project{anax}{github.com/miguelmartin75/anax}
        {Personal Project}{2013 - Present}
        {
            An open source entity system written in C++11, as a portable library.
            \begin{itemize}
                \item 142 stars on GitHub
            \end{itemize}
        }
        \project{Virtual Robot}{bitbucket.com/miguelmartin/virtual-robot}
        {Personal Project}{2011}
        {
            An educational piece of software aimed to teach the basics of programming,
            written for my electronics class in grade 10.
            \begin{itemize}
                \item Written in Java
                \item Multi-threaded Java Swing application
                \item Uses the MVC pattern
            \end{itemize}
        }
        \project{pine}{github.com/miguelmartin75/pine}
        {Personal Project}{2013 - 2014}
        {
            Pine is a general, lightweight, header-only C++11 library, which is designed to make organisation of a game much simpler. 
            \begin{itemize}
                \item Further enhanced my API design skills
                \item Increased knowledge of C++ templates
                \item Heavily documented
            \end{itemize}
        }
        Please visit my "projects" section on my website for more projects and details: miguel-martin.com/projects.
    }
}

\info{Education}
{
    \textbf{University of Adelaide}\newline
    \emph{Bachelor of Computer Science (Advanced) \hfill 2014 - Present}
    \begin{description}[leftmargin=0.5cm, itemindent=0cm]
        \item Expected Graduation Year: 2016.
        \item GPA: 6.0/7.0
        \item Received University of Adelaide Principals' Scholarship in 2014.
        \item Key Courses:
            \begin{itemize}
                \item Object Oriented Programming\dotfill High Distinction
                \item Algorithm Design and Data Structures\dotfill Distinction
                \item Mathematics IA\dotfill Distinction
                \item Mathematics IB\dotfill Distinction
                \item Systems Programming in C\dotfill TBA
                \item Computer Networks and Applications\dotfill TBA
            \end{itemize}
    \end{description}
}

\info{Technology Summary}
{
    \begin{multicols}{3}
    \subsection*{Languages}
    {
        \begin{description}
            \item C++\dotfill 3 years
            \item Java\dotfill 2.5 years
            \item C\#\dotfill 2 years
            \item Python\dotfill 0.5 years
            \item Objective-C\dotfill 0.5 years
        \end{description}
    }

    \columnbreak

    \subsection*{Version Control}
    {
        \begin{description}
            \item Git\dotfill 2 years
            \item SVN\dotfill 1 year
            \item Mercurial\dotfill 0.5 years
        \end{description}
    }

    \columnbreak

    \subsection*{Tools}
    {
        \begin{description}
            \item Visual Studio\dotfill 2 years
            \item Vim\dotfill 2 years
            \item Xcode\dotfill 2 years
            \item Make\dotfill 1.5 years
            \item CMake\dotfill 1 year
        \end{description}
    }
    \end{multicols}
}

\end{document}
